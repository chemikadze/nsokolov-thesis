%\chapter{Биологическое воздействие высокочастотного излучения на организм человека}
\chapter{БИОЛОГИЧЕСКОЕ ВОЗДЕЙСТВИЕ ВЫСОКОЧАСТОТНОГО ИЗЛУЧЕНИЯ НА ОРГАНИЗМ ЧЕЛОВЕКА}

Электромагнитные волны возникают при ускоренном движении электрических зарядов. 
Электромагнитные волны, или электромагнитное излучение –- это взаимосвязанное 
распространение в пространстве изменяющихся электрического и магнитного полей. 
Совокупность этих полей, неразрывно связанных друг с другом, называется электромагнитным 
полем. К высокочастотному электромагнитному излучению относят волны частотой от 3 МГц до 300 ГГц.\cite{2011ecology}

Источники высокочастотного излучения в природе практически не встречаются, и в естественной
среде человек не испытывает на себе его влияния.

С другой стороны, высокочастотное излучение получило широкое использование в самых
различных областях: телекоммуникациях, навигации, компьютерной технике и бытовой технике.
Без использования высоких частот трудно представить современное радио, телевидение,
сотовую связь и даже обычную квартиру -- источники этого излучения окружают человека 
повсеместно.

Еще на ранних этапах использования было замечено, что высокочастотные волны имеют
вредное воздействие на организм человека. Даже непродолжительное влияние волн
достаточной интенсивности может стать причиной головных болей, повышенной утомляемости, 
нарушений сна. Продолжительное воздействие может привести к серьезным последствиям,
вплоть до повреждения сердца, мозга, центральной нервной системы.

Поле высокой частоты обладает высокой проникающей способностью, вызывает колебания ионов, 
электронную и атомную поляризацию -- смещение электронных оболочек и атомных групп в 
пределах молекулы, а также ориентационную или дипольную поляризацию в полярных молекулах.
Поглощенная энергия поля УВЧ преобразуется в тепловую, а также оказывает характерное для 
высокочастотных полей нетепловое -- осцилляторное действие.
Особенностью УВЧ является избирательное прогревание тканей организма соответственно их 
физико-химическим свойствам, способности к рассеиванию тепла, которая зависит от 
кровоснабжения и теплопроводности. Так как излучение влияет в большей степени на температуру
внутренних органов, и в меньшей степени -- на температуру кожных покровов, человек,
испытывающий на себе вредное воздействие не может его заметить, что представляет собой
отдельную опасность.

Более чувствительными к действию электромагнитных полей являются центральная нервная система и нейроэндокринная система.

С нарушением нейроэндокринной регуляции связывают эффект со стороны сердечно-сосудистой 
системы, системы крови, иммунитета, обменных действий и других систем организма. 
Влияние на иммунную систему выражается в изменении свойств сыворотки крови, нарушении 
белкового обмена, угнетении Т-лимфоцитов. Возможны также изменение частоты пульса, 
сосудистых реакций. 
Описаны конфигурации кроветворения, нарушения со стороны эндокринной системы, 
метаболических действий, заболевания органов зрения. 

Среди всех клинических проявлений радиоволн выделяют на три синдрома: астенический, 
астеновегетативный и гипоталамический.

Астенический синдром обычно наблюдается в начальных стадиях заболевания и проявляется 
жалобами на головную боль, завышенную утомляемость, раздражительность, нарушение сна, 
периодически возникающие боли в области сердца.

Астеновегетативный  синдром характеризуется реакциями кровеносной системы, такими как 
гипотония, брадикардия и другие.

Гипоталамический синдром же действует в основном на нервную систему, вызывая повышенную 
возбудимость, неустойчивость. В отдельных вариантах обнаруживаются признаки раннего 
атеросклероза, ишемической болезни сердца, гипертонической болезни.

Поля сверхвысоких частот могут оказывать действие на глаза, приводящее к возникновению 
катаракты (помутнению хрусталика), а умеренных -- к изменению сетчатки глаза по типу. 

Предполагается, что нарушение регуляции физиологических функций организма обусловлено 
действием поля на разные отделы нервной системы. При этом повышение возбудимости 
центральной нервной системы происходит за счет рефлекторного деяния поля, а тормозной 
эффект -- за счет прямого действия поля на структуры головного и спинного мозга. Считается, что 
кора головного мозга, а также промежуточный мозг в особенности чувствительны к действию поля. 
В последние годы возникают сообщения о способности индукции ЭМИ злокачественных 
заболеваний. Еще немногочисленные данные все же говорят, что наибольшее число случаев 
приходится на опухоли кроветворных тканей и на лейкоз в частности. Это становится общей 
закономерностью канцерогенного эффекта при воздействии на организм человека и животных 
физических факторов различной природы и в ряде остальных случаев.

В настоящий момент нормы воздействий высокочастотного излучения регламентированы
в СанПиН 2.2.4.1191-03 "Электромагнитные поля в производственных условиях".
\footnote{Утверждены Главным государственным санитарным врачом Российской Федерации 30.01.2003, действуют с 1 мая 2003 года}\cite{2003sanpin}.

В соответствии с этим документом, оценка и нормирование электромагнитного поля в диапазоне 
частот от 30 кГц до 300 ГГц осуществляется по величине энергетической экспозиции (ЭЭ).
При этом в диапазоне от 30 кГц до 300 МГц ЭЭ расчитывается по формулам:
\begin{eqnarray}
    ЭЭ_{H} &=& H^2 \times T  \\
    ЭЭ_{E} &=& E^2 \times T  
\end{eqnarray}
где $Е$ -- напряженность электрического поля ($В/м$), $Н$ -- напряженность магнитного поля ($А/м$), $Т$ -- время воздействия за смену (час.).

Для диапазона частот выше 300 МГц ЭЭ расчитывается по формуле:
\begin{equation}
  ЭЭ_{ППЭ} = ППЭ \times T
\end{equation}
где $ППЭ$ -- плотность потока энергии ($Вт/м^2$, $мкВт/см^2$).

Соответственно, 
для частот от 3 МГц до 30 МГц установлено ПДУ для $ЭЭ_{E}$ в $7000 (В/м)^2\times ч$
для частот от 30 МГц до 300 МГц установлено ПДУ для $ЭЭ_{E}$ в $800 (В/м)^2\times ч$,
а для частот выше 300 МГц -- $ЭЭ_{ППЭ}$ в $200 (мкВт/см^2) \times ч$.

Кроме этого, для регламентируются предельно допустимые уровни напряженности и плотности потока энергии
в соответствующих диапазонах.
Для диапазона частот от 3 до 30 МГц предельно допустимый уровень напряженности 
электрического поля составляет $300 В/м$, для частот от 30 до 300 МГц -- $80 В/м$.

Для диапазона частот выше 300 МГц устанавливается предельно допустимый уровень ППЭ $1000 мкВт/см^2$ и в случае локального облучения кистей рук -- $5000 мкВт/см^2$.

При этом требуется, чтобы соблюдение требований на рабочих местах должно осуществляться 
как на этапах, предшествующих началу эксплуатации рабочего помещения, так и после,
при организации новых рабочих мест, их аттестации, а так же в порядке текущего надзора
за источниками электромагнитного излучения.

Контроль может проводиться двумя методами: расчетным и путем измерений.
При этом расчетные методы используются преимущественно при проектировании новых или реконструкции действующих объектов, а для действующих объектов контроль осуществляется преимущественно посредством инструментальных измерений. Для оценки уровней 
излучения могут использоваться приборы направленного и ненаправленного приема.

При проведении измерений необходимо, чтобы работники отсутствовали на рабочих местах
во время контроля, а источник работал с максимальной мощностью. При этом запрещено
производить такие измерения при наличии атмосферных осадков а так же любых 
условиях, выходящих за предельные рабочие параметры средств измерений.
Сам инструмент должен иметь свидетельство о государственной аттестации.
В конечном итоге, по результатам измерений составляется протокол или карта распределения 
уровня электромагнитных полей.

Стоит отметить, что СанПиН позволяет не контролировать источники, мощность излучения 
которых меньше определенного уровня для конкретной частоты.
Кроме этого, при измерениях учитывается поза человека, наличие нескольких источников
излучения и постоянство излучения во времени. Благодаря большому количеству
учтенных деталей, полученные результаты составляют достаточно достоверную картину 
о дозах излучения, которые получает человек в ходе стандартной рабочей смены.

Для уменьшения вреда, наносимого высокочастотным излучением, может проводиться целый 
ряд защитных мер. Защитные меры можно разделить на два класса: организационные, 
инженерно-технические и лечебно-профилактические.

К организационным мерам можно отнести:
\begin{itemize}
    \item выбор рациональных режимов работы оборудования;
    \item выделение зон воздействия излучения разной интенсивности;
    \item расположение рабочих мест и маршрутов передвижения обслуживающего персонала на 
    расстояниях от источников излучения;
    \item проведение ремонта оборудования, являющегося источником излучения, вне зоны влияния 
    излучения других источников;
    \item соблюдение правил безопасной эксплуатации источников.
\end{itemize}

Инженерно-технические мероприятия проводятся с целью обеспечения более низких
уровней электромагнитного излучения на рабочих местах. Достижение этой цели может 
проходить как путем внедрения новых технологий, так и применения средств коллективной и индивидуальной защиты.

Последний вид мер, лечебно-профилактические, имеет целью предупреждение и раннее
обнаружение изменений состояния здоровья всех лиц, профессионально связанных с 
обслуживанием и эксплуатацией источников излучения. Для этого необходимо проводить
предварительные и периодические профилактические медосмотры.

Кроме этого, для отдельного круга лиц используются более низкие предельно допустимые
нормы, равные ПДУ для населения. В этот список входят лица, не достигшие 18-летнего возраста, и женщины в состоянии беременности.

Как видно, работа в условиях повышенного уровня высокочастотного электромагнитного
излучения требует повышенных требований к работодателю и работникам, в связи с опасной
природой этого явления. К счастью, эта область достаточно хорошо изучена и регламентирована
соответствующими контролирующими документами, поэтому сохранение жизни и здоровья
работников не должно быть проблемой при правильной организации.
