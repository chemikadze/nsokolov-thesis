\intro

Концепция виртуализации сама по себе не нова, и использовалась IBM в их мейнфреймах
с середины 60-х годов. Тем не менее, эта технология снова привлекла к себе внимание
в связи с последними успехами в создании многоядерных процессоров с аппаратной
поддержкой виртуализации, улучшенными механизмами доступа к памяти и 
программным обеспечением для управления виртуализованными ресурсами.

Виртуализация -- это абстракция между ресурсами с которыми работает пользователь
и реальными физическими ресурсами, при которой с точки зрения пользователя 
не проявляется явная связь его ресурсов с физическими.\cite{Carapinha:2009:NVV:1592648.1592660} 
В последние годы подобные технологии
стали широко применяться в промышленности, в основном благодаря усилиям крупных
игроков индустрии.\cite{website:oracle-vt}\cite{website:microsoft-vt}\cite{website:redhat-vt}

Возможности современного аппаратного обеспечения позволяют получить огромные
вычислительные возможности в одной машине -- десятки процессорных ядер и десятки гигабайт оперативной памяти. Зачастую, эти мощности являются избыточными для большинства
применений. Тем временем перед многими предприятиями стоит двоякая задача
одновременно поддерживать большое количество отдельных информационных систем, 
и при этом уменьшить сложность и стоимость поддержки этой инфраструктуры.

Современным решением этой проблемы является использование виртуальных машин.
Несколько виртуальных машин, будучи запущенными на одном физическом узле, 
с одной стороны являются изолированными друг от друга, а с другой стороны используют
общие ресурсы, что позволяет использовать аппаратное обеспечение более оптимальным образом.
При этом, каждая такая машина работает под управлением своей собственной операционной 
системы, которая никак не связана с операционной системой физического сервера.
Такими образом, каждая машина может быть независимым образом включена, выключена,
и даже приостановлена независимо от остальных. Каждая виртуальная машина кроме этого 
имеет независимые виртуальные диск, сетевую карту, видеокарту и другого оборудования.
Все эти ресурсы управляются специальным программным обеспечением, называемым 
гипервизором, запущенным на реальной физической машине.
\cite{Kamoun:2009:VDW:1595422.1595424}

% TODO: исправить сокращения
Интерес к виртуализации сети среди специалистов в сфере сетевых технологий заметно
увеличился за последнее время. Идея виртуализации сети сама по себе не нова, и использовалась
ранее в виде виртуальных частных сетей (VPN). Этот подход позволял успешно использовать
независимые виртуальные сети поверх общей физической инфраструктуры. Тем не менее,
у имеющихся реализаций этого подхода имеются свои ограничения, такие как:
\begin{itemize}
    \item Все виртуальные сети в пределах одной физической инфраструктуры должны базироваться
    на одном стеке протоколов, что ограничивает возможности по совместному использованию
    общей инфраструктуры информационными системами, основанными на разных технологиях.
%    \item Реальная изоляция виртуальных сетевых ресурсов не возможна.
     \item Невозможно разделить источник инфраструктуры и источник виртуальных сетей,
     так как они управляются одной сущностью.
% TODO: нипанятна :(
\end{itemize}
Виртуализация сетей заходит дальше, позволяя иметь виртуальные сети, работающие по
независимым протоколам. К примеру это означает, что виртуальная сеть не обязательно
должна базироваться на стеке протоколов IP или какой-то другой технологии, что 
дает гораздо большую гибкость при выборе и построении виртуальной инфраструктуры ИС.
% TODO: есть куда улучшать

Другим преимуществом виртуализации сетей является естественная возможность работы
в разделяемом окружении, скрывая существование внутреннего административного деления.
Несмотря на то, что подобные решения существуют и для виртуальных частных сетей,
эта проблема всегда была трудно решаемой в таких случаях.
\cite{Carapinha:2009:NVV:1592648.1592660}

Современная виртуализация часто связана с так называемыми облачными вычислениями.
Этот термин в общем означает, что и приложения, и вычислительные ресурсы предоставляются в виде услуг через интернет.\cite{Armbrust:2010:VCC:1721654.1721672} Современное деление предполагает существование трех категорий
поставщиков таких услуг:\cite{Creeger:2009:CCO:1551644.1554608}
\begin{description}
    \item[IaaS] Предоставление вычислительных мощностей в централизованном виде
    \item[PaaS] Базовые элементы для создания приложений
    \item[SaaS] Приложения, доступные через глобальные компьютерные сети.
\end{description}

Несмотря на тесную связь необходимо различать облачные вычисления и виртуализацию вообще. 
Первое лишь является парадигмой вычисления, операционной моделью, которая позволяет
предоставлять динамически масштабируемые разделяемые ресурсы по требованию, в виде услуг.
Несмотря на то, что виртуализация не является необходимым условием создания облачных
систем, она играет главную роль в практической реализации подобных систем, помогая
снизить простои физического оборудования, занимаемые площади и потребление
электроэнергии. Это основная причина, по которой виртуализация получает такое широкое 
распространение.
\cite{Kamoun:2009:VDW:1595422.1595424}

Рассматривая виртуализацию сетей с в контексте облачных вычислений, можно выделить следующие преимущества:
\begin{itemize}
    \item С точки зрения поставщика, становится возможным предоставить
    каждому пользователю индивидуальную виртуальную сеть с интересующими пользователя     
    топологией, параметрами, стеком протоколов, гарантиями на обслуживание.
    \item Для пользователя это дает более широкие возможности в выборе как самих технологий,
    так и вариантах конфигурации.
\end{itemize}

Стоит отметить, что виртуализация сетей имеет большой потенциал при проведении 
исследований, в том числе для проведения испытаний новых технологий, или изучении
конкретных сценариев использования уже существующих. В случае использования
сторонних поставщиков виртуальной инфраструктуры, отпадает необходимость в покупке
дорогостоящего оборудования для лабораторий, так как все необходимые ресурсы можно
получать по требованию.

При построении подобной инфраструктуры внутри рабочей группы или организации, 
использование этой техники так же дает положительный результат -- одна инфраструктура
может использоваться в разное время для разных задач. Таким образом, количество
необходимого оборудования уменьшается в разы.

В данной работе пойдет речь о разработке программного комплекса, ядра информационной 
системы, предоставляющей запуск виртуальных топологий сетей TCP/IP. 
В каком-то смысле, данная система
является гибридной -- с одной стороны, имеется возможность запуска виртуальных машин,
с другой же стороны, в отличии от имеющихся аналогов, необходима абсолютная гибкость
в конфигурации сети, соединяющей эти машины. Кроме этого ставится задача виртуализации
не только обычных IBM PC-совместимого аппаратного обеспечения, но и сетевого оборудования,
такого как коммутаторы и маршрутизаторы.

В первой главе будет дан обзор существующих технологий, используемых для виртуализации
и создания информационных систем, предоставляющих инфраструктуру как сервис. 
% TODO: и что там я напишу еще

Во второй главе будет описана архитектура разработанной системы, ее компоненты, требования
к ним и их характеристики. Кроме этого, в ней приведен анализ возможных будущих 
применений данной системы.

В третьей главе пойдет речь о варианте конкретной реализации программного комплекса на 
основе разработанной архитектуры, а так же описаны варианты использования полученного 
прототипа и пути его развития.

В четвертой главе будет рассмотрен косвенный экономический эффект, оценена
себестоимость и значимость данной работы.

В пятой главе будут рассмотрены вопросы безопасности жизнедеятельности, в частности,
биологическое воздействие высокочастотного излучения на организм человека.




