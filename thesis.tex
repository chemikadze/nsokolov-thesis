\documentclass[%
specialist,     % тип документа
subf,           % подключить и настроить пакет subfig для вложенной нумерации рисунков
href,           % подключить и настроить пакет hyperref
colorlinks=true % цветные гиперссылки
%,times         % шрифт Times как основной
%,fixint=false  % отключить прямые знаки интегралов
]{disser}

\usepackage[
  a4paper, mag=1000,
  left=2.5cm, right=1cm, top=2cm, bottom=2cm, headsep=0.7cm, footskip=1cm
]{geometry}
\usepackage{hyperref}
\usepackage{url}
\usepackage{mathtext}
\usepackage[T2A]{fontenc}
\usepackage[utf8]{inputenc}
\usepackage[english,russian]{babel}
\ifpdf\usepackage{epstopdf}\fi

% Номера страниц снизу и по центру
%\pagestyle{footcenter}
%\chapterpagestyle{footcenter}

% Точка с запятой в качестве разделителя между номерами цитирований
%\setcitestyle{semicolon}

% Использовать полужирное начертание для векторов
\let\vec=\mathbf

% Включать подсекции в оглавление
\setcounter{tocdepth}{2}

%----------------------------------------------------------------
\begin{document}

%
% Титульный лист на русском языке
%

% Название организации
\institution{Московский государственный университет геодезии и картографии}

% Имя лица, допускающего к защите (зав. кафедрой)
\apname{к.\,т.\,н.\, доцент А. В. Кононов}

\title{Дипломная работа}

% Тема
\topic{Разработка платформы виртуализации и моделирования сетей TCP/IP на базе OpenStack и dynamips}

% Группа
\group{Студента группы ИСИТ V-1с}
% Автор
\author{Соколов Николай Валерьевич}

% Научный руководитель
\sa       {И.\,Н.~Кондауров}
\sastatus {к.\,т.\,н., доцент}

% Рецензент
% \rev      {П.\,П.~Петров}
% \revstatus{к.\,ф.-м.\,н., доцент}

% Город и год
\city{Москва}
\date{\number\year}

\maketitle

% Содержание
\tableofcontents

% Введение
\intro

Концепция виртуализации сама по себе не нова, и использовалась IBM в их мейнфреймах
с середины 60-х годов. Тем не менее, эта технология снова привлекла к себе внимание
в связи с последними успехами в создании многоядерных процессоров с аппаратной
поддержкой виртуализации, улучшенными механизмами доступа к памяти и 
программным обеспечением для управления виртуализованными ресурсами.

Виртуализация -- это абстракция между ресурсами с которыми работает пользователь
и реальными физическими ресурсами, при которой с точки зрения пользователя 
не проявляется явная связь его ресурсов с физическими. В последние годы подобные технологии
стали широко применяться в промышленности, в основном благодаря усилиям крупных
игроков индустрии \cite{website:oracle-vt}\cite{website:microsoft-vt}\cite{website:redhat-vt}.
\cite{Carapinha:2009:NVV:1592648.1592660}

Возможности современного аппаратного обеспечения позволяют получить огромные
вычислительные возможности в одной машине -- десятки процессорных ядер и десятки гигабайт оперативной памяти. Зачастую, эти мощности являются избыточными для большинства
применений. Тем временем перед многими предприятиями стоит двоякая задача
одновременно поддерживать большое количество отдельных информационных систем, 
и при этом уменьшить сложность и стоимость поддержки этой инфраструктуры.

Современным решением этой проблемы является использование виртуальных машин.
Несколько виртуальных машин, будучи запущенными на одном физическом узле, 
с одной стороны являются изолированными друг от друга, а с другой стороны используют
общие ресурсы, что позволяет использовать аппаратное обеспечение более оптимальным образом.
При этом, каждая такая машина работает под управлением своей собственной операционной 
системы, которая никак не связана с операционной системой физического сервера.
Такими образом, каждая машина может быть независимым образом включена, выключена,
и даже приостановлена независимо от остальных. Каждая виртуальная машина кроме этого 
имеет независимые виртуальные диск, сетевую карту, видеокарту и другого оборудования.
Все эти ресурсы управляются специальным программным обеспечением, называемым 
гипервизором, запущенным на реальной физической машине.
\cite{Kamoun:2009:VDW:1595422.1595424}

% TODO: исправить сокращения
Интерес к виртуализации сети среди специалистов в сфере сетевых технологий заметно
увеличился за последнее время. Идея виртуализации сети сама по себе не нова, и использовалась
ранее в виде виртуальных частных сетей (VPN). Этот подход позволял успешно использовать
независимые виртуальные сети поверх общей физической инфраструктуры. Тем не менее,
у имеющихся реализаций этого подхода имеются свои ограничения, такие как:
\begin{itemize}
    \item Все виртуальные сети в пределах одной физической инфраструктуры должны базироваться
    на одном стеке протоколов, что ограничивает возможности по совместному использованию
    общей инфраструктуры информационными системами, основанными на разных технологиях.
%    \item Реальная изоляция виртуальных сетевых ресурсов не возможна.
     \item Невозможно разделить источник инфраструктуры и источник виртуальных сетей,
     так как они управляются одной сущностью.
% TODO: нипанятна :(
\end{itemize}
Виртуализация сетей заходит дальше, позволяя иметь виртуальные сети, работающие по
независимым протоколам. К примеру это означает, что виртуальная сеть не обязательно
должна базироваться на стеке протоколов IP или какой-то другой технологии, что 
дает гораздо большую гибкость при выборе и построении виртуальной инфраструктуры ИС.
% TODO: есть куда улучшать

Другим преимуществом виртуализации сетей является естественная возможность работы
в разделяемом окружении, скрывая существование внутреннего административного деления.
Несмотря на то, что подобные решения существуют и для виртуальных частных сетей,
эта проблема всегда была трудно решаемой в таких случаях.
\cite{Carapinha:2009:NVV:1592648.1592660}

Современная виртуализация часто связана с так называемыми облачными вычислениями.
Этот термин в общем означает, что и приложения, и вычислительные ресурсы предоставляются в виде услуг через интернет.\cite{Armbrust:2010:VCC:1721654.1721672} Современное деление предполагает существование трех категорий
поставщиков таких услуг:\cite{Creeger:2009:CCO:1551644.1554608}
\begin{description}
    \item[IaaS] Предоставление вычислительных мощностей в централизованном виде
    \item[PaaS] Базовые элементы для создания приложений
    \item[SaaS] Приложения, доступные через глобальные компьютерные сети.
\end{description}

Несмотря на тесную связь необходимо различать облачные вычисления и виртуализацию вообще. 
Первое лишь является парадигмой вычисления, операционной моделью, которая позволяет
предоставлять динамически масштабируемые разделяемые ресурсы по требованию, в виде услуг.
Несмотря на то, что виртуализация не является необходимым условием создания облачных
систем, она играет главную роль в практической реализации подобных систем, помогая
снизить простои физического оборудования, занимаемые площади и потребление
электроэнергии. Это основная причина, по которой виртуализация получает такое широкое 
распространение.
\cite{Kamoun:2009:VDW:1595422.1595424}

Рассматривая виртуализацию сетей с в контексте облачных вычислений, можно выделить следующие преимущества:
\begin{itemize}
    \item С точки зрения поставщика, становится возможным предоставить
    каждому пользователю индивидуальную виртуальную сеть с интересующими пользователя     
    топологией, параметрами, стеком протоколов, гарантиями на обслуживание.
    \item Для пользователя это дает более широкие возможности в выборе как самих технологий,
    так и вариантах конфигурации.
\end{itemize}

Стоит отметить, что виртуализация сетей имеет большой потенциал при проведении 
исследований, в том числе для проведения испытаний новых технологий, или изучении
конкретных сценариев использования уже существующих. В случае использования
сторонних поставщиков виртуальной инфраструктуры, отпадает необходимость в покупке
дорогостоящего оборудования для лабораторий, так как все необходимые ресурсы можно
получать по требованию.

При построении подобной инфраструктуры внутри рабочей группы или организации, 
использование этой техники так же дает положительный результат -- одна инфраструктура
может использоваться в разное время для разных задач. Таким образом, количество
необходимого оборудования уменьшается в разы.

В данной работе пойдет речь о разработке программного комплекса, ядра информационной 
системы, предоставляющей запуск виртуальных топологий сетей TCP/IP. 
В каком-то смысле, данная система
является гибридной -- с одной стороны, имеется возможность запуска виртуальных машин,
с другой же стороны, в отличии от имеющихся аналогов, необходима абсолютная гибкость
в конфигурации сети, соединяющей эти машины. Кроме этого ставится задача виртуализации
не только обычных IBM PC-совместимого аппаратного обеспечения, но и сетевого оборудования,
такого как коммутаторы и маршрутизаторы.

В первой главе будет дан обзор существующих технологий, используемых для виртуализации
и создания информационных систем, предоставляющих инфраструктуру как сервис. 
% TODO: и что там я напишу еще

Во второй главе будет описана архитектура разработанной системы, ее компоненты, требования
к ним и их характеристики. Кроме этого, в ней приведен анализ возможных будущих 
применений данной системы.

В третьей главе пойдет речь о варианте конкретной реализации программного комплекса на 
основе разработанной архитектуры, а так же описаны варианты использования полученного 
прототипа и пути его развития.

В четвертой главе будет рассмотрен косвенный экономический эффект, оценена
себестоимость и значимость данной работы.

В пятой главе будут рассмотрены вопросы безопасности жизнедеятельности, в частности,
биологическое воздействие высокочастотного излучения на организм человека.






\chapter{ОБЗОР ТЕХНОЛОГИЙ}
%\chapter{Обзор технологий}

\section{Виртуализация IBM-PC совместимого оборудования}

\subsection{XEN}

% этой секции нужна ревизия -- очень путанно и непонятно

В традиционных гипервизорах виртуальное аппаратное обеспечение представлено в 
виде, функционально идентичном реальному оборудованию, что позволяет запускать
внутри виртуальных машин немодифицированные версии операционных систем. Тем не менее,
у этого способа есть и свои минусы, особенно при эмуляции архитектуры x86. К примеру,
определенные инструкции машинных кодов должны перехватываться гипервизором
и выполняться особым образом. Так же, эффективная реализация блока управления памятью
(MMU) так же представляет отдельную сложность. Несмотря на трудности, эти и другие 
проблемы могут быть решены, но ценой этому будут сложность реализации и производительность.

Кроме специфичных для x86 архитектур проблем, против полной виртуализации существуют
и другие контраргументы. В частности, зачастую желательно, чтобы гостевая операционная
система видела не только виртуальные, но и реальные ресурсы. К примеру, при решении
задач, требовательных к точности временных отсчетов, может быть полезно предоставлять
в гостевой операционной системе реальное время наряду с внутренним виртуальным.

Чтобы избежать всех этих проблем, в XEN используется абстракция над реальным аппаратным
обеспечением, называемая паравиртуализацией. Позволяя достичь большей производительности,
эта техника тем не менее требует некоторых модификаций в гостевой операционной системе.
При этом стоит обратить внимание, что изменения в операционной системе не затрагивают
прикладного двоичного интерфейса (application binary interface, ABI), так что в гостевые
приложения в модификации не нуждаются.

% TODO: что такое домен

Вместо эмуляции реального аппаратного обеспечения, XEN предоставляет для операционной
системы набор простых абстракций над устройствами, что позволяет достичь большей
эффективности. В частности, весь ввод-вывод между доменами происходит через разделяемую
память при помощи асинхронных колец дескрипторов буфера. С одной стороны это позволяет
достичь больших скоростей при передаче данных, а с другой -- проводить различные 
проверки безопасности на стороне гипервизора, к примеру, что дескриптор буфера указывает
на реальную область памяти домена, от которого поступил запрос на запись.

Через всю архитектуру XEN проходит принцип отделения политик от механизмов. Несмотря на
то, что гипервизор обязан участвовать во всех низкоуровневых операциях вроде планировки
процессорного времени между виртуальными машинами, фильтровке сетевых пакетов перед
отправкой или проверок прав доступа при чтении данных, совершенно не требуется, чтобы 
в нем была реализована логика распределения процессорного времени или правила фильтрации
пакетов.

В результате, сам XEN предоставляет только набор самых базовых управляющих операций.
Доступ к этим операциям имеется только из специальных авторизованных доменов. Все 
вышеназванные правила при этом выполняются специальной управляющей программой, 
запущенной в таком управляющем домене, а не превилегированным кодом гипервизора.

Стоит обратить внимание, что домен, работающий с управляющим интерфейсом запускается
в момент загрузки XEN-хоста. Этот домен, называемый Domain0, как раз и содержит в себе
управляющие программы. Так же, программы запущенные в нем предоставляют набор операций
по манипуляции другими доменами, доступ к конифгурационным параметрам и политикам.

Кроме управления ресурсами процессора и памяти, управляющий интерфейс поддерживает
создание и удаление виртуальных сетевых интерфейсов (virtual network interfaces -- VIFs) и 
блочных устройств (virtual block devices -- VBDs). С этими виртуальные устройствами ввода-вывода
связывается набор политик, в которых указан набор доменов, которые имеют к ним доступ,
а так же какие операции над ними позволено производить.
\cite{Barham:2003:XAV:1165389.945462}

\subsection{QEMU}

QEMU -- пример классической полной виртуализации; основывается на базе динамического
транслятора. Он эмулирует несколько архитектур процессора: x86, PowerPC, ARM и SPARC на
большом количестве физического аппаратного обеспечения: x86, PowerPC, ARM, SPARC, Alpha
и MIPS, причем целевая архитектура и физическая архитектура могут не совпадать. К сожалению,
в силу вышеперечисленных возможностей, выполнение кода внутри данного гипервизора
замедляется в среднем от 4 до 10 раз.\cite{Bellard:2005:QFP:1247360.1247401}

Логически в QEMU можно выделить несколько основных подсистем:
\begin{itemize}
    \item Эмулятор процессора
    \item Эмулированные устройства (VGA-экран, последовательный порт, клавиатура и мышь PS/2,
     жесткий диск IDE, сетевая карта NE2000)
    \item Обычные устройства (блочные, символьные, сетевые) для соединения эмулятора с 
    реальными устройствами на хостовой машине.
\end{itemize}
Стоит отметить, что благодаря вышеперечисленному делению на подсистемы и открытым кодам,
QEMU послужил основой для ряда производных работ.
\cite{Raghav:2012:FSS:2159430.2159442} 
\cite{Becker:2012:XEQ:2380356.2380368} 
\cite{Hong:2012:HMR:2259016.2259030}
\cite{Ding:2011:PPS:2117686.2118470}
\cite{Nakamoto:2009:PUE:1524877.1525243}
% TODO: перечислить

Так как QEMU представляет реализует полную виртуализацию, на нем могут быть запущены
не модифицированные операционные системы со стандартным набором драйверов.
Сам QEMU работает на трех наиболее популярных операционных системах: Windows, 
Linux и Mac OS X.

Этот гипервизор может применяться для запуска приложений, предназначенных для 
операционной системы, отличной от ОС хост-машины, для тестирования поведения
программ, а так же в качестве "песочницы" при работе с недоверенным кодом.

Кроме этого, интересной особенностью этого гипервизора является поддержка в 
режиме эмулятора процессора. Этот режим является своеобразным подмножеством полного
эмулятора, позволяя запускать процессы из бинарных файлов архитектур, отличных от 
архитектуры хост-машины в контексте хостовой операционной системы. Одним из применений 
этого режима является проверка результата работы кросс-компиляторов без необходимости
запускать виртуальную машину с полной копией операционной системы.

\subsection{KVM}

KVM (Kernel Virtual Machine) -- относительно новый среди представленных гипервизоров,
относительно простой, но тем не менее обладающий очень высокими характеристиками. 
KVM, так же как и QEMU,
является реализацией полной виртуализации, но при этом полностью использует возможности 
аппаратного ускорения виртуализации современных процессоров.
\cite{Habib:2008:VK:1344209.1344217}

Разработчики KVM, вместо реализации больших частей операционных систем в своем
гипервизоре, как сделали разработчики других гипервизоров, нашли способ как заставить
работать ядро Linux в качестве гипервизора, что было достигнуто путем реализации
KVM в виде модуля ядра. Интеграция возможностей по виртуализации в ядро позволило
упростить работу с ресурсами в гипервизоре, а так же улучшить производительность в целом.
 
Данный подход имеет множество преимуществ. Используя ядро Linux в качестве основы для
своей работы, разработчики смогли избавить себя от создания таких подсистем, как
планировщик задач, так как в данной реализации виртуальная машина представлена в системе
обычным процессом. Достичь этого удалось с помощью объявления нового режима исполнения 
кода. Традиционно Linux использует два режима: ядра и пользователя, KVM же добавляет 
промежуточный гостевой режим.
 
Типичная установка KVM состоит из следующих компонент:

\begin{itemize}
    \item Драйвер для управления виртуальными устройствами, управление которым доступно
    через символьное устройство /dev/kvm.
    \item Программа для управления виртуальным аппаратным обеспечением, как правило
    специальной версией QEMU.
    \item Модулем ввода вывода, как правило эти возможности так же полностью предоставлены
    QEMU.
\end{itemize}

Интересной возможностью KVM, которая реализована благодаря полному доступу гипервизора
к внутренним механизмам ядра, является объединение одинаковых страниц 
(Kernel Same-page Merging -- KSM). KSM сканирует память виртуальных машин на предмет
одинаковых страниц памяти, и объединяет их в одну, так что одна и та же область памяти
используется несколькими виртуальными машинами. Если гость пытается изменить такую 
разделяемую область памяти, то для него будет вновь создана уникальная копия.

% TODO: здесь можно написать про то, как я делал ето и сколько оно дает

Используя эту функцию KVM в случаях, когда запускается много виртуальных
машин с одинаковой версией операционной системы, можно достичь большой экономии 
памяти, так как в данном случае создается
большое количество страниц с одинаковым содержимым, в котором находится в основном
неизменяемый бинарный код.

Еще одним преимуществом реализации KVM в виде модуля ядра Linux является превосходная
поддержка большого количества аппаратного обеспечения, так как все устройства, 
поддерживаемые данной операционной системой, могут быть использованы совместно
с гипервизором.
\cite{RedHat:kvm}



\section{Моделирование сети}

\subsection{ns2}

При исследованиях в области компьютерных сетей симуляции играют большую роль, так как
имеют много преимуществ по сравнению с проверками на реальном железе.
Сетевые симуляторы позволяют реализовать и изучать различные сетевые сущности в 
смоделированном окружении, что особенно хорошо при исследовании новых протоколов,
технологий, физических моделей и топологий.

% TODO: что такое дискретный?
NS-2 -- это дискретный сетевой симулятор, который продолжительное время был стандартом
де-факто при академических исследованиях.

В NS-2 широко используется язык Tcl, так как на момент создания этого симулятора
компиляция C++ файлов требовала относительно много времени. C++ использовался для
основных элементов и моделей, которые относительно стабильные и не требуют частых
изменений. Интерпретируемый язык Tcl же использовался для описания сценариев 
экспериментов, позволяя избежать долгой фазы компиляции.
\cite{Font:2010:ADS:1878537.1878651}

На настоящий момент имеющееся в доступе аппаратное обеспечение обеспечивает
адекватное время компиляции, поэтому преимущества от использования языка Tcl 
практически сошли на нет.

NS-2 разделяет структуру предметной области на сетевую топологию и агентов, которые
обмениваются данными через нее. Топология образована узлами и ребрами между ними,
причем с ребрами могут быть связаны объекты, которые описывают поведение соединения
между узлами, а так же характеристики канала и другие параметры.
Агенты являются по сути сетевыми приложения, которые запущены на узлах. Разработчик
модели сам выбирает формат данных, и при необходимости может зарегистрировать
в NS-2 свой собственный формат.

Архитектуру NS-2 можно разделить на 5 частей:
\begin{enumerate}
    \item Планировщик событий
    \item Сетевые компоненты, такие как узел и ребро
    \item Tcl
    \item OTcl -- объектное расширение к Tcl
    \item TclCL -- обертку вокруг OTcl для работы с объектами из C++
\end{enumerate}

Моделирование NS-2 можно классифицировать как дискретно-событийное.
Дискретно-событийное моделирование -- это вид имитационного моделирования, при 
котором функционирование системы представляется как хронологическая последовательность 
событий. Событие происходит в определенный момент времени и знаменует собой изменение 
состояния системы.

Планировщик предназначен для управления выполнением симуляции, и отвечает за выполнение
событий в определенные моменты времени.
Планировщики в NS-2 подразделяются на планировщики реального времени и
планировщики виртуального времени (non-real-time schedulers). Планировщики реального
времени используются при эмуляции сети, когда симулятор взаимодействует с реальной
внешней сетью.

Известной проблемой NS-2, к которая послужила причиной его устаревания является
необходимость редактирования C++ кода самого симулятора при разработке новых форматов
данных и протоколов. Так как эта проблема была заложена в самой архитектуре симулятора, 
единственным методом ее решения явилось полное переписывание.

\subsection{ns3}

NS-3 является дискретно-событийным симулятором для сетевых исследований,
и продолжает родословную NS-2. В отличии от предшественника, в котором все симуляции
описывались на языке Tcl, в данном случае их можно описывать как на C++, так и на Python.
Так как NS-3 был полностью написан с нуля, старые симуляции NS-2 не могут быть на нем
запущены.


При создании NS-3 разработчики выбрали целью улучшить следующие показатели NS-2:
\begin{itemize}
    \item модульность компонент
    \item масштабируемость симуляций
    \item интеграция со сторонним кодом
    \item поддержка эмуляции
    \item трассировка и статистика
    \item валидация
\end{itemize}

Так как большое количество исследований требует расширения возможностей
симулятора, NS-3 поддерживает расширение базовой функциональности путем
наследования базовых классов из стандартной реализации, при этом не требуя 
полной перекомпиляции самого симулятора.

Так же, благодаря правильному проектированию, NS-3 позволяет повторно использовать
части моделей, написанные для других симуляций. В основном это достигнуто за счет
того, что большая часть общего кода написана в терминах указателей на общие базовые классы,
из-за чего возможно совместное использование стандартных и сторонних модулей без 
изменения основного кода.

Другой новой функций NS-3 является трассировка, написанная с использованием
функций обратного вызова, что позволяет отделить код трассировки от общего кода.
В качестве формата данных используется libpcap, так как существует много уже существующих
утилит для работы с ним.
% TODO: источник?

Еще одним новшеством NS-3 является лучшая масштабируемость симуляций. Среди используемых
техник можно перечислить распределенные симуляций на базе протоколов PDNS и GTNetS,
кэширование результатов большой вычислительной емкости и более гибкую инфраструктуру
трассировки.

\subsection{dynamips}

Несмотря на несомненную полезность сетевых симуляторов в академических исследованиях
сетевых протоколов, они, тем не менее, не позволяют моделировать реально существующие
модели оборудования и особенности реализаций стеков протоколов различных операционных
систем. Кроме этого, даже будучи заложенной в симулирующую программу, эта логика
может иметь отличия от реального прообраза.
Решить эти проблемы может только виртуализация оборудования.

% TODO: источник
Одним из наиболее популярных вендоров сетевого оборудования является компания Cisco,
известная так же своими сертификационными экзаменами. Зачастую у готовящихся
к экзамену нет доступа к реальному оборудованию, и долгое время единственными решениями
этой проблемы было либо прохождение дорогостоящих курсов в учебных центрах или 
использование специального ПО для симуляций, что так же не всегда удобно по 
вышеперечисленным причинам.

Эмулятор Dynamips\cite{website:dynamips} используется для виртуализации маршрутизаторов и 
коммутаторов фирмы Cisco. 
Dynamips позволяет загружать реальные образы Cisco Internetworking Operating 
System (Cisco IOS) и проводить
настройку виртуального аппаратного обеспечения таким же образом, как и реального.
\cite{Zhang:2011:CCV:1975507.1976866}

Dynamips может работать в двух режимах -- обычном и режиме гипервизора. В обычном режиме
каждая виртуальная машина запускается в своем собственном процессе, и не может быть
сконфигурирована в ходе работы. Режим же гипервизора позволяет запускать несколько 
виртуальных машин в одном процессе и производить конфигурацию виртуальной машины 
в ходе работы виртуального устройства.

Для связи портов виртуальной машины с внешним миром Dynamips предоставляет абстракцию
NIO (Network Input-Output), позволяя использовать различные способы для установления
соединения между виртуальными машинами или даже реальными интерфейсами реального
физического оборудования. На данный момент поддерживаются следующие виды:
\begin{itemize}
    \item UNIX сокеты
    \item User Mode Linux и Virtual Distributed Ethernet  % TODO ссылка
    \item Виртуальные сетевые интерфейсы linux (tap-устройства)
    \item Реальные сетевые интерфейсы
    \item Инкапсуляция в UDP-каналы
    \item Инкапсуляция в TCP-каналы
\end{itemize}

К сожалению, динамическая трансляция, которую выполняет dynamips, накладывает свои
ограничения на производительность виртуальных сетевых устройств. Кроме того,
некоторые модели оборудования Cisco не могут быть в нем сэмулированы, так как представляют
из себя композицию нескольких параллельно работающих устройств.

Тем не менее, для того ряда устройств, которые официально поддерживаются, обеспечена
полная работоспособность всех функций, доступных на реальном оборудовании. К ним
относятся:
\begin{itemize}
    \item 7000
    \item 3600 (3620, 3640 и 3660)
    \item 3700 (3725 и 3745)
    \item 2600 (от 2610 до 2650XM, 2691)
\end{itemize}

Типичный процесс работы выглядит следующим образом.
\begin{enumerate}
    \item Пользователь запускает несколько виртуальных машин dynamips, указывая тип
    эмулируемого оборудования и способ их соединения. К примеру, может использоваться
    UDP NIO, для чего необходимо выбрать номера портов для каждого соединяемого
    виртуального порта каждого устройства, и индивидуально для каждого порта указать
    локальный и удаленный адрес соединения.
    Так же необходимо указать номер порта, на котором будет доступна виртуальная консоль.
    \item Dynamips загружает образ операционной системы IOS, запускает его, и инициализирует
    NIO на указанных портах, в данном случае создавая UDP-сокеты с переданными параметрами.
    \item Пользователь подключается с помощью программы удаленного терминала (например,
    telnet или putty) и производит работу с консолью IOS.
\end{enumerate}

Так как процесс настройки NIO достаточно трудоемок и требует внимательности, для облегчения
этого была написана программа Dynagen.\cite{website:dynagen} Она состоит из двух частей:
низкоуровневой реализации API гипервизора dynamips на языке python и высокоуровневого
модуля, который использует файлы конфигурации с относительно простым синтаксисом
для задания взаимосвязей между устройствами.

Тем не менее не для всех может быть удобен процесс работы с файлами конфигурации, 
поэтому существует еще один, более высокоуровневый проект -- GNS 3.\cite{website:gns3}
Он представляет собой простой графический интерфейс, написанный на python и Qt, и использует
для запуска топологий Dynagen.

\section{Облачные системы}

%\subsection{Eucalyptus}
%Eucalyptus первоначально был создан в качестве платформы для так называемых "частных
%облаков". С самого начала он разрабатывался полностью совместимым с Amazon EC2 API,
%чтобы пользователи имели возможность мигрировать от Amazon в свое приватное 
%облако и наоборот без переписывания кода.
%
%Eucalyptus состоит из пяти компонентов:
%\begin{enumerate}
%    \item Контроллер облака (Cloud Controller -- CLC), который управляет виртуализованными ресурсами
%    \item Контроллер кластера (Cluster Controller -- CC), который управляет запуском виртуальных машин
%    \item Система хранения данных Walrus 
%    \item Контроллер хранения данных (Storage Controller -- SC), предоставляющий блочное     
%    хранилище
%    \item Контроллер узлов (Node Controller -- NC), специальная служба устанавливаемая на
%    все узлы, предназначенные для запуска виртуальных машин
%\end{enumerate}
%Эти компоненты распределены по вычислительным узлам системы, и конкретная топология
%зависит от конкретного случая.
%
%Eucalyptus имеет несколько режимов работы сети: управляемая (managed), 
%управляемая без LAN (managed noLAN), системная и статическая. В первых двух случаях
%пользователь управляет сетью вручную, с той разницей что в первом случае сети изолированы
%с помощью vLAN. В системном режиме виртуальные машины получают адреса от внешнего 
%DHCP-сервера. В статическом режиме адреса выдаются DHCP-сервером, управляемым из 
%Eucalyptus.
%
%\cite{vonLaszewski:2012:CMC:2353730.2353779}
%
%\subsection{OpenNebula}
%
%OpenNebula -- еще один проект по созданию IaaS-платформы с открытым исходным кодом.
%Данный проект отличается гибкостью и простотой, и при этом хорошей 
%функциональностью.
%
%Кроме этого, OpenNebula может работать в распределенном окружении, включающем
%в себя несколько установок системы в различных местах, благодаря чему можно легко
%создавать системы, устойчивые к отказам в каком-либо конкретном датацентре. При этом,
%работая в таком режиме, все подсистемы-участники управляются единообразно из одной точки
%доступа.\cite{vonLaszewski:2012:CMC:2353730.2353779}
%
%% поддержка API и распределения
%
%Простота данной системы достигнута ценой введения единой точки отказа. Все управление
%вычислительными ресурсами ведется с помощью одной программы, запущенной на 
%управляющем узле. При этом все управление ресурсами ведется с помощью протокола ssh.
%
%Работа с дисками виртуальных машин так же отличается простотой и гибкостью. Поддерживаются
%следующие режимы:
%\begin{itemize}
%    \item Диски хранятся как обычные файлы на локальных дисках, образа для запуска копируются вручную на вычислительные узлы.
%    \item Диски хранятся как файлы в разделяемой распределенной ФС (к примеру, NFS).
%    \item Диски являются разделами LVM (Linux Volume Manager).
%\end{itemize}
%
%Имеется так же три режима работы с сетью:
%\begin{itemize}
%   \item изоляция с помощью vLAN
%   \item изоляция с помощью ebtables (фаервол виртуальных сетевых мостов Linux)
%   \item использование программного коммутатора Open vSwitch
%\end{itemize}

\subsection{OpenStack}
OpenStack -- относительно молодой проект с открытым исходным кодом, 
объединяющий платформу для запуска виртуальных машин Nova, хранилище 
данных Swift, хранилище образов дисков Glance и несколько других проектов.
\cite{vonLaszewski:2012:CMC:2353730.2353779}

Начало проекта было дано хостинг-провайдером Rackspace и NASA, когда в июле 2010 года
были открыты исходные коды Rackspace Cloud Files -- родоначальника проекта Swift, 
и внутренних разработок NASA в сфере IaaS -- родоначальника проекта Nova.
Со временем к проекту подключилось множество других компаний, в числе которых
Canonical, Cisco, HP и Intel.

OpenStack Nova подразделяется на следующие компоненты:
\begin{itemize}
    \item nova-api -- REST-интерфейс для управления виртуальными машинами
    \item nova-scheduler -- планировщик ресурсов
    \item nova-compute -- сервис, управляющий виртуальными машинами на узлах
    \item nova-network -- менеджер сетей
    \item nova-volume -- предоставляет блочное хранилище для виртуальных машин
    \item nova-consoleauth, nova-novncproxy и другие -- удаленный доступ по протоколу VNC
\end{itemize}
Эти компоненты связаны с помощью сервиса обмена сообщений RabbitMQ, и используют
для хранения данных SQL-совместимые СУБД при помощи ORM-библиотеки SQLAlchemy.

Данный проект является хорошим примером модульного дизайна. 
Каждый сервис состоит из двух частей: обертка для поддержки распределенных вызовов,
которая является единой для всех сервисов, и класса-менеджера, реализующего логику работы.
При этом для каждого сервиса реализацию менеджера легко подменить, указав необходимое
имя класса в конфигурационном файле.

Следующим уровнем абстракции являются так называемые драйвера. Некоторые менеджеры,
такие как nova-compute и nova-scheduler, с одной стороны должны поддерживать достаточно 
большое количество логики, которая не представляют интереса при конфигурации.
С другой стороны часть логики может иметь достаточно много альтернативных реализаций,
к примеру различные алгоритмы распределения виртуальных машин по узлам или
поддержка новых гипервизоров. В таких случаях общая логика реализуется в менеджере,
а специфичные функции выделяются в драйвер. Конкретная реализация драйвера, как и в случае
с менеджером, указывается в конфигурационном файле.

Работа с OpenStack выглядит следующим образом:
\begin{itemize}
    \item Пользователь при помощи утилиты glance загружает специально подготовленный 
    образ диска для виртуальной машины, получая при этом идентификатор диска.
    \item При помощи утилиты nova, пользователь запускает нужную ему виртуальную машину,
    указывая идентификатор диска, так называемый flavor -- набор параметров, описывающий
    характеристики виртуальной машины, такие как количество виртуальных процессоров, 
    количество ОЗУ и ПЗУ, публичную часть SSH-ключа для работы с этой машиной и некоторые
    другие параметры.
    \item Запрос на запуск обрабатывается nova-api, и отправляется в nova-scheduler.
    \item nova-scheduler определяет узел, который может быть использован для запуска 
    виртуальной машины с полученными параметрами, и передает запрос в nova-compute
    на соответствующем узле.
    \item nova-compute получает образ диска от glance
    \item nova-compute запрашивает сетевые ресурсы у nova-network
    \item nova-compute запускает виртуальную машину в гипервизоре, передавая ему все
    полученные параметры, включая файл образа диска, IP-адрес и SSH-ключ.
    \item Пользователь получает IP-адрес виртуальная машина. 
    Для работы с машиной могут быть использованы протоколы SSH и VNC.
    \item Когда машина больше не нужна пользователю, он может отправить запрос в nova-api, 
    который будет передан nova-compute. В результате этого сетевые ресурсы будут освобождены,
    виртуальная машина остановлена, а виртуальный диск -- удален.
\end{itemize}

Недавним нововведением в платформе является сервис виртуальных сетей Quantum, который 
пришел на смену nova-network. Этот сервис предоставляет REST-интерфейс для управления
примитивами канального и сетевого уровня модели OSI. Реализация виртуальных сетей
поверх какой-либо реальной технологии носит название плагина. 
В стандартной поставке имеются плагины, реализующие виртуальные сети при помощи
следующих технологий:
\begin{itemize}
    \item виртуальные сетевые мосты Linux
    \item оборудование Cisco
    \item программный коммутатор Open vSwitch
\end{itemize}

Типичная реализация плагина состоит из нескольких частей:
\begin{itemize}
    \item Менеджер сетей, реализующий настройку оборудования и хранение данных.
    \item Плагин API, предоставляющий специфическую информацию о ресурсах. 
    Эта информация используется в дальнейшем клиентами для подключения к сети.
    \item Клиентский код, который производит локальные настройки для подключения к сети.
    При этом используется информация от API-плагина для получения нужных параметров.
\end{itemize}









\chapter{Архитектура системы}

инструментарий-действия-теория-технология. выводы.
\chapter{РАЗРАБОТКА МОДУЛЕЙ}
% \chapter{Разработка модулей}

\section{Выбор стека технологий}

Перед началом реализации разработанной архитектуры необходимо выбрать
стек технологий, которые будут использоваться при разработке.
Необходимо определиться со следующими пунктами:
\begin{itemize}
    \item операционная система
    \item гипервизоры
    \item реализация управляющей логики
    \item API
\end{itemize}
Касательно гипервизоров, необходимо выбрать способ виртуализации обычных виртуальных машин
и сетевого оборудования.

В качестве операционной системы предпочтительнее всего использовать Linux, так как ее 
администрирование не представляет собой большой проблемы, она бесплатна и имеет
открытый исходный код и отличную поддержку большинства современных языков 
программирования. Кроме этого, для администрирования Linux существуют доказавшие
свою эффективность системы автоматизации установки и администрирования, что совершенно
необходимо для успешной поддержки многомашинных систем.
%TODO источники

Если говорить о виртуализации сетевого оборудования, то имеется два пути: использовать
специализированные гипервизоры, или же использовать стандартные гипервизоры вместе с
программными коммутаторами, такими как Open vSwitch. На текущий момент, программные
коммутаторы редко используются при создании сетей в предприятиях, поэтому большого
смысла поддерживать этот вариант нет. По данным Infonetics Research, Cisco занимает
лидирующие позиции на рынке сетевого оборудования,
% http://www.infonetics.com/pr/2012/3Q12-Service-Provider-Routers-Switches-Market-Highlights.asp
поэтому выбор этого сетевого оборудования в качестве цели для виртуализации вполне оправдан.
Кроме этого, гипервизор dynamips, предназначенный для эмуляции именно этого оборудования,
является самым распространенным и стабильным в своей области. %TODO источник

Среди гипервизоров PC-совместмого оборудования можно выделить несколько наиболее
популярных решений для Linux:
\begin{itemize}
    \item VmWare ESX
    \item VirtualBox
    \item Xen
    \item KVM
\end{itemize}

VmWare ESX является одним из наиболее популярных коммерческих гипервизоров, но при 
использовании в рамках данного проекта имеет ряд недостатков. Во-первых, гипервизор
работает только со специальным образом модифицированным ядром ОС, во-вторых,
данный гипервизор имеет закрытый код, и в случае необходимости становится невозможным
внести изменения, необходимые для реализации проекта.

VirtualBox -- гипервизор с открытым исходным кодом, в данный момент поддерживаемый 
компанией Oracle. Данный продукт в основном базируется как решение для настольных
компьютеров, несмотря на возможность работы без графического интерфейса.
Ранее VirtualBox разрабатывался компанией Sun, которая была куплена Oracle в 2010 году.
Переход многих программ с открытым исходным кодом во владение к Oracle сказался на
их поддержке крайне негативно, поэтому в долгосрочной перспективе использование
VirtualBox при реализации данного проекта не имеет большого смысла при наличии альтернатив.

Xen долгое время был выбором по-умолчанию в Linux-среде и активно поддерживался
самой большой Linux-компанией -- RedHat. %TODO источник
Тем не менее, последние несколько лет интерес к нему падает, так как
RedHat переключил свой фокус на KVM. Так же, в силу того, что этот гипервизор 
использует технику паравиртуализации, его администрирование имеет ряд отличий
от администрирования обычных Linux-систем.

KVM -- относительно новый, но уже достаточно популярный и стабильный Linux-специфичный
гипервизор. На данный момент он активно поддерживается, в том числе коммерческими компаниями,
такими как RedHat, и на настоящий момент является выбором по-умолчанию для виртуализации в 
Linux. Кроме этого, благодаря открытому коду, в него будет возможно внести изменения,
если такая необходимость возникнет во время разработки.
Таким образом, в качестве гипервизора PC-совместимых виртуальных машин в данном проекте
будет использоваться KVM.

При разработка управляющей логики имеется два варианта: писать весь необходимый код с нуля, 
используя только стандартные библиотеки языков, или же расширить возможности имеющейся
платформы. Написание кода с нуля с одной стороны позволяет получить полный контроль 
над кодовой базой и выбором технологий, но с другой стороны -- потребует реализации
большого объема относительно стандартной логики, которая не имеет большого отношения 
к сути данной работы. Гораздо более оптимальным вариантом является использование уже 
существующих наработок.

Как нельзя кстати в данном случае подходит платформа OpenStack. Рассматривая компоненты,
из которых она состоит, можно отметить, что архитектура этой платформы хорошо вписывается
в архитектуру, разработанную в предыдущей главе. Соответствие компонент OpenStack и 
предполагаемых компонент платформы моделирования сети можно увидеть на рис.~\ref{fig:openstack-lowlevel}.
\begin{figure}
  \centering
  {\footnotesize\input{fig/openstack-lowlevel}}
  \caption{Соответствие архитектуры OpenStack и архитектуры системы моделирования топологий}  
  \label{fig:openstack-lowlevel}
\end{figure}
Очевидно, что большинство компонент OpenStack может быть использовано в 
системе моделирования сетей либо напрямую, без всякой модификации, либо путем
написания соответствующего расширяющего кода. Список компонентов, подлежащих
повторному использованию, представлен в таблице~\ref{tab:openstack-reuse}
\begin{table}
\center
\caption{Возможность посторного использования компонент OpenStack}
\label{tab:openstack-reuse}
\begin{tabular}{|p{5cm}|p{4cm}|p{5cm}|} \hline 
\multicolumn{3}{|c|}{подсистема прикладного программного интерфейса} \\ \hline 
авторизация и аутентификация & quantum & без изменений\\ \hline
логика высокоуровневых команд & - & необходима реализация \\ \hline
\multicolumn{3}{|c|}{подсистема хранения данных} \\ \hline
хранение образов ОС & glance & без изменений \\ \hline
хранение данных виртуальных машин & на узлах nova-compute или в nova-volume & без изменений\\ \hline
\multicolumn{3}{|c|}{вычислительная подсистема} \\ \hline
подсистема планировки ресурсов & nova-scheduler & без изменений \\ \hline
подсистема сетевого взаимодействия & quantum & необходим дополнительный модуль \\ \hline
подсистема виртуализации & nova-compute & необходим дополнительный модуль \\ \hline
\hline 
\end{tabular} 
\end{table}

Путем использования уже существующей платформы, при реализации системы моделирования
сетей мы избегаем реализации большого количества не относящейся к сути логики. В итоге мы
получаем три основных модуля, требующих реализации:
\begin{enumerate}
    \item высокоуровневый прикладной программный интерфейс
    \item сетевое взаимодействие
    \item запуск специфических виртуальных машин -- маршрутизаторов в гипервизоре dynamips
\end{enumerate}

Прикладной программный интерфейс должен работать в терминах разработанной модели данных.
Следуя концепции модульной архитектуры, он будет реализован в виде отдельного процесса,
и будет транслировать высокоуровневые вызовы пользователя в ряд вызовов к нижележащим
подсистемам: nova-api и quantum.
%TODO технологии?

Модель сети, которая используется внутри OpenStack по-умолчанию устроена так, что 
все виртуальные машины одного проекта находятся в одной подсети. Тем не менее,
используя последние наработки сообщества разработчиков этой платформы, возможно использование
quantum для гибкой настройки соединения между виртуальными машинами.

Для того, чтобы конфигурировать сеть между виртуальными машинами в OpenStack на 
канальном уровне, необходимо реализовать два аспекта. Во-первых, необходим такой способ 
инкапсуляции кадров канального уровня, который позволит запускать достаточно большое
количество виртуальных машин в одной системе, обеспечивая изоляцию между ними.
Во вторых, необходимо предоставлять данные о физических портах в подсистему виртуализации.

Проекты GNS3 и dynagen решают проблему инкапсуляции при помощи использования UDP-каналов.
Для работы dynamips совместно с KVM при данном подходе ранее было необходимо использование 
специально модифицированной версии Qemu, но на настоящий момент UDP-каналы
поддерживаются и в официальной версии эмулятора.

Проблема предоставления данных о физических портах в систему виртуализации вытекает
из ограниченной сетевой модели OpenStack, которая использовалась с самого начала разработки
проекта. Вместо явного указания количества виртуальных сетевых адаптеров и способа их 
соединения с сетями, в nova-compute передавался только список сетей, к которым необходимо
подключить виртуальную машину. По этой причине не существует стандартного способа передачи
таких метаданных о подключенной сети, как номер сетевой карты и порта.
Для решения проблемы метаданных портов в рамках этого проекта создано расширение прикладного 
программного интерфейса Quantum, позволяющее задавать и получать метаданные порта.

Архитектура nova-compute изначально была расчитана на поддержку различных гипервизоров.
Благодаря этому, для поддержки виртуализации сетевого оборудования системой моделирования
сетей, необходимо создать так называемый драйвер гипервизора. Интерфейс драйвера гипервизора
содержит около 60 методов, из которых далеко не все обязательны для реализации, и
в нашем случае задача сводится к созданию простой обертки над dynagen, которая конфигурирует
виртуальные маршрутизаторы согласно информации, полученной от прикладного программного
интерфейса и расширения Quantum для метаданных.

\section{Описание прикладного программного интерфейса}
Интерфейс программирования приложений (иногда интерфейс прикладного программирования) (англ. application programming interface, API) --  набор готовых классов, процедур, функций, структур и констант, предоставляемых приложением (библиотекой, сервисом) для использования во внешних программных продуктах.

API определяет функциональность, которую предоставляет программа (модуль, библиотека), при этом абстрагируя конкретную реализацию этой функциональности. Благодаря API возможно создание
программ, взаимодействующих с другими программами.

Так как в рамках данной работы делается акцент на разработку логики системы моделирования сетей,
разработанная реализация не имеет графического пользовательского интерфейса. Вместо этого,
система предоставляет свой собственный API для взаимодействия, оперирующий в терминах 
модели предметной области.

На настоящий момент, наиболее часто используемым способом организации взаимодействия
программ по сети является REST. REST (Representational State Transfer, «передача представлений 
состояний») не имеет в основе конкретного стандарта, и является не более чем стилем построения 
архитектуры распределенного приложения. Принципы REST были впервые описаны в диссертации
одного из разработчиков протокола HTTP, Роя Филдинга.

Как правило, под REST сервисами подразумеваются сервисы, которые предоставляют пользователю
доступ к некоторому набору ресурсов, с которыми можно взаимодействовать посредством
HTTP-запросов. При этом ресурсы кодируются в одном или нескольких возможных заранее
определенных форматах, как правило человекочитаемых, таких как XML или JSON.
В данном случае в качестве формата используется JSON, так как он является более компактным
и легкочитаемым по сравнению с XML, а так же наиболее распространенным на данный момент.

Несмотря на то, что сам по себе REST не является стандартом, существуют стандарты на описание
интерфейсов REST-сервисов. Если для представления данных используется JSON, то можно
применить стандарты JSON Schema и JSON Hyper-Schema. Первый стандарт позволяет
регламентировать формат представления ресурсов: имена полей, типы данных и т.д. Второй
стандарт расширяет JSON Schema, позволяя задавать ссылки на связанные ресурсы, а так же
ссылки на действия над данным ресурсом.

OpenStack использует для аутентификации сервис Keystone. Перед работой с системой, пользователь
должен сначала получить так называемый токен у Keystone, который необходимо отправлять в 
заголовке HTTP-запроса \verb`HTTP_X_AUTH_TOKEN` при обращении к Nova, Glance и другим сервисам. 
В целях унификации, прикладной программный интерфейс системы моделирования так же будет 
использовать Keystone для своей работы.

Как и в OpenStack, пользователи системы сгруппированы в группы, называемые тенантами 
(англ. tenants). Так как с одной стороны, разные группы пользователей в перспективе могут иметь 
разные картотеки сущностей, а с другой стороны REST подразумевает, что содержимое 
ресурса полностью описывается его URL, то принято решение ко всем адресам сущностей добавить
в виде префикса идентификатор группы.

При выделении сущностей в ресурсы необходимо поддерживать баланс между гранулярностью
и денормализованностью. Слишком большая гранулярность делает описание API раздутым и сложным,
а слишком денормализованное -- увеличивает количество передаваемых данных, и зачастую 
требует большого количества кода для извлечения желаемых сущностей из представления ресурса.

Анализируя предполагаемую работу с системой, можно выделить следующие наиболее частые действия:
\begin{itemize}
   \item создание, получение и удаление топологии
   \item получение списка типов узлов для запуска новой топологии со стандартным 
            набором сетевых адаптеров для узла
   \item получение списка типов сетевых адаптеров для запуска новой топологии, в которой
            к узлам подключены сетевые адаптеры, отличные от конфигурации по-умолчанию
\end{itemize}

Таким образом, можно выделить три ресурса:
\begin{itemize}
    \item типы узлов
    \item типы сетевых адаптеров
    \item топологии
\end{itemize}
Первые два должны изменяться только административно, последние же -- самим пользователем.

\subsection{Типы сетевых адаптеров}

%TODO реализовать параметры в коде
Тип сетевого адаптера имеет три атрибута: имя $name$ и список портов $ports$. 
Доступные типы сетевых адаптеров представлены
в виде единого ресурса, благодаря чему при редактировании топологии можно единожды
запросить список адаптеров и использовать его в дальнейшем.  Для упрощения кода на стороне 
пользователя, множество доступных адаптеров задается в виде объекта JSON с ключами,
равными имени типа адаптера. URL для ресурса будет иметь вид "/{tenantid}/cards", ресурс не изменяем
для пользователей. 

JSON-схема ресурса, таким образом, принимает следующий вид:

\lstinputlisting{schema/schema_slots.json}

Что соответствует подобному JSON-документу:

\lstinputlisting{code/slots.json}


\subsection{Типы узлов}

%TODO реализовать метаданные, параметры и os
%TODO исчез id из слотов
Типы узлов имеют следующие аттрибуты: идентификатор $id$, человекочитаемое название
$name$, список портов для подключения сетевых адаптеров $slots$, 
набор параметров $parameters$, список доступных для запуска версий операционных систем 
$software$, а так же набор метаданных $metadata$. Сетевые адаптеры
по-умолчанию указаны напрямую в списке портов.
Так же, как и типы сетевых адаптеров, типы представлены в виде единого ресурса, объединенные в 
объект JSON. URL ресурса -- "/{tenantid}/devices", для пользователя этот ресурс доступен только для чтения.

JSON-схема типов узлов имеет следующий вид:

\lstinputlisting{schema/schema_hardware.json}

Что соответствует следующему JSON-документу:

\lstinputlisting{code/hardware.json}


\subsection{Топологии}

%TODO добавить софт
%TODO добавить параметры
%TODO добавить URL'ы консолей

Топологию, как было рассмотрено ранее, можно разделить на список узлов и список 
связей. При этом каждый узел имеет идентификатор $id$, человекочитаемое название
$name$, список слотов с подключенными адаптерами $slots$, 
набор значений параметров $parameters$, версию операционной системы $software$, 
а так же набор метаданных $metadata$. 

В целях упрощения реализации будем считать, что каждая связь может иметь только два 
подключенных порта  $left$ и $right$, и не имеет каких-либо специфических параметров.
Так же, каждое соединение имеет идентификатор связи $id$.

Топологии являются единственными ресурсами, которые может редактировать пользователь.
Они доступны по URL "/{tenantid}/instances/{id}", и могут быть созданы, прочтены с сервера и удалены.
Так же, у топологии существует дочерние ресурсы -- веб-консоли для устройств,
содержащие единственное поле адреса, и доступные по URL вида "/{tenantid}/instances/{id}/consoles/{device}".
В виде JSON-схемы вышеописанные требования принимают следующий вид:

\lstinputlisting{schema/schema_instances.json}

Что соответствует следующему JSON-документу:

\lstinputlisting{code/instances.json}

%TODO коды возврата HTTP

\section{Подсистема прикладного программного интерфейса}
Подсистема прикладного программного интерфейса системы моделирования достаточно проста,
основная ее задача -- по описанию топологии из запроса отправить некоторое количество
запросов сервисам OpenStack для настройки сети и вычислительных узлов.
При этом на стороне прикладного программного интерфейса нет никакой необходимости 
иметь какую-либо логику для обращения к Keystone в целях проверки аутентификации пользователя,
так как эту проверку будут выполнять сервисы OpenStack, получающие такие же заголовки
аутентификации, что и сам прикладной программный интерфейс.

В случае ошибки при выполнении любого запроса, результат успешно завершившихся запросов
будет откачен, а пользователь получит соответствующий код ошибки и краткое текстовое объяснение.

В подсистеме прикладного программного интерфейса можно выделить следующие компоненты:
\begin{itemize}
    \item контроллер действий
    \item отображение внутреннего представления во внешнее
    \item логика управления топологиями
    \item хранение моделей топологий
\end{itemize}

Контроллер содержит в себе логику обработки запроса и реализован при помощи микрофреймворка
Flask. При использовании данной технологии, обработчики запросов являются функциями в модуле
со специальными аннотациями. К примеру, получение списка запущенных топологий реализовано
следующим образом:
\begin{lstlisting}
@app.route("/<tenant>/instances", methods=["GET"])
def instance_list(tenant):
    instances = 
        app.facade.get_instances(RequestContext(request))
    return api_response(
        status=200, 
        payload=map(instance_to_json, instances))
\end{lstlisting}
Из полученного запроса извлекается OpenStack-специфичный контекст, который передается
в фасад, реализующий логику управления топологиями (в данном случае -- получение списка 
топологий). При этом модели топологий отображаются во внешнее представление
при помощи функции \verb`instance_to_json`. Логика отображения представлений достаточно
проста и не требует пояснений, так как сводится к формированию одних JSON-документов из других.

Логика управления топологиями предоставляет следующие операции:
\begin{itemize}
    \item \verb`launch_instance` (запуск топологии)
    \item \verb`destroy_instance` (остановка топологии)
    \item \verb`update_instance_status` (обновление статуса отдельных узлов)
    \item \verb`get_instance` (получение хранимого представления топологии)
    \item \verb`get_instances` (получение всех хранимых топологий)
    \item \verb`get_network_cards` (получение всех типов сетевых адаптеров)
    \item \verb`get_device_types` (получение всех типов узлов)
\end{itemize}

Операции над топологиями выполняют несколько запросов к сервисам OpenStack, по одному для 
каждого узла и каждой связи.
В качестве примера рассмотрим псевдокод запуска топологии:
\begin{lstlisting}
def launch_instance(self, ctx, instance):
  conn_info = DeviceConnections()
  created_networks = []
  created_devices = []
  try:
    for wire in instance['wires']:
      net = create_network(wire)
      created_networks.append(net)
      lport = create_port(instance, net, wire)
      rport = create_port(instance, net, wire)
      conn_info.add(wire['left'], lport)
      conn_info.add(wire['right'], rport)
    for device in instance['devices']:
      conn = conn_info.get(device['id'])
      device = create_device(device, conn)
      created_devices.append(device)
  except Exception as e:
    delete_devices_ignoring_errors(created_devices)
    delete_networks_ignoring_errors(created_devices)
    raise
  db.create_instance(instance)
  return instance
\end{lstlisting}
Объект \verb`conn_info` используется только для хранения промежуточной информации о 
подключениях.
Вся логика сводится к созданию соответствующих соединениям сетей в Quantum, и запуску
узлов в OpenStack, подключенных к соответствующим сетям. В случае любых ошибок
созданные устройства и сети удаляются.

Операция \verb`get_network_cards` не требуют запросов к OpenStack, и использует для выполнения 
список сетевых карт, предоставляемый библиотекой dynamips. 

%TODO фильтрация типов по флейворам
%TODO фильтация операционных систем
Операция получения всех типов узлов \verb`get_device_types`, в свою очередь,
так же использует dynamips для получения списка виртуального сетевого оборудования.
При этом список поддерживаемого оборудования фильтруется по списку типов виртуальных
машин OpenStack -- так называемых флейворов. Таким образом из поддерживаемого
dynamips оборудования в выдачу попадают только те, которые имеют соответствующую запись
в списке флейворов OpenStack. Формат флейворов для виртуального сетевого оборудования 
имеет вид \verb`r1.{model}`, где \verb`model` -- название модели устройства, к примеру 
\verb`c2610`.

Кроме этого, операция получения списка типов узлов совершает запрос к glance для того, чтобы 
найти список поддерживаемых операционных систем. Для каждого сетевого устройств будут указаны
только те операционные системы, чье название содержит идентификатор платформы
типа устройства, к примеру \verb`c2600` для модели \verb`c2610`.

Те флейворы, которые начинаются с префикса, отличного от \verb`r1`, к примеру
\verb`m1.small`, считаются обычными PC-совместимыми виртуальными машинами. 
При этом, в качестве образов операционных систем отображаются все хранимые образы, 
чье имя не совпало с названием ни с одной платформы.

\section{Подсистема сетевого взаимодействия}
TBD

\section{Подсистема виртуализации}
TBD

\section{Физическая топология}
TBD



\chapter{Организационно-экономическое обоснование}

\section{Характеристика организации работ}

Разработка платформы виртуализации и моделирования сетей TCP/IP на базе OpenStack и dynamips выполнена на кафедре информационно-измерительных систем под руководством кандидата технических наук доцента Кондаурова И.\,Н. 

Исходные данные: 
\begin{itemize}
   \item исходные тексты платформы виртуализации OpenStack
   \item исходные тексты гипервизора dynamips
   \item исходные тексты модуля dynagen
   \item исходные текста программы GNS 3
\end{itemize}

Приборное и программное обеспечение:
\begin{itemize}
    \item ноутбук Mac Book Air mid 2011 (Core i7 1.7 ГГц, 4 Гб ОЗУ, 128 Гб ПЗУ)
    \item среда разработки IntelliJ IDEA 11
    \item программное обеспечение виртуализации аппаратного обеспечения маршрутизаторов, dynamips
\end{itemize}

Целью работы является разработка архитектуры программного обеспечения, позволяющей проводить моделирование сетей TCP/IP.

В результате работы разработан набор дополнительных модулей, позволяющих 
использовать платформу виртуализации OpenStack для моделирования топологий сетей TCP/IP.

\section{Обоснование косвеного экономического эффекта}

Полученный в результате данной работы набор модулей может быть использован как 
непосредственно для запуска виртуализированных топологий путем подготовки специально
оформленных файлов описания, так и как часть более сложной системы, скрывающей от
пользователя конкретный синтаксис описания, в том числе с помощью графического 
интерфейса.

Построенный на базе этих модулей программно-аппаратный комплекс может быть использован
для произведения:
\begin{itemize}
    \item предварительного моделирования при создании реальной сети TCP/IP
    \item лабораторных работ, связанных с исследованием различных сетевых технологий и протоколов
    \item испытаний программного обеспечения, построенного по распределенным архитектурам, в том числе одноранговой и клиент-серверной
\end{itemize}

Стоит отметить, что все вышеуказанные способы применения данного продукта могут 
осуществляться как локально, при расположении пользователей и ИС, построенной с 
использованием разработанных модулей в пределах одного помещения, так и при 
существенном их удалении. Таким образом, результаты данной работы могут применяться 
при создании систем дистанционного обучения.

Кроме этого, одним из очевидных путей дальнейшего развития данной технологии 
является построение системы для предоставления платформы виртуализации сетей 
в виде услуги через сеть Интернет.

Использование вышеописанной платформы так же имеет положительный экономический
эффект и для конечного пользователя, позволяя арендовывать по требованию необходимые
для моделирования вычислительные ресурсы. К примеру, проведение 
предварительных испытаний топологий сетей требуется лишь на определенных этапах
разработки ИС: проведении экспериментов для выбора наилучшей топологии 
сети распределенной системы, подборе конкретных параметров перед внедрением.
Но покупка необходимого оборудования для проведения подобных испытаний
не совсем целесообразна, так как это оборудование не будет участвовать в дальнейшем
при разработке и сопровождении ИС.

\section{Расчет себестоимости дипломной работы}

Себестоимостю данной дипломной работы является сумма всех затрат, имевших место
при подготовке этой работы.

Себестоимость разработки дипломного проекта можно выразить в виде:
\begin{equation}
    C = C_{o} + C_{к} + C_{t} + C_{p} \\
\end{equation}
где $C$ -- общая себестоимость, $C_{к}$ -- издержки, связанные с работой на компьютере
и в сети интернет, $C_o$ -- издержки, связанные с оплатой труда, $C_t$ -- издержки на транспорт и $C_p$ -- прочие издержки.

Издержки, связанные с оплатой труда представлены в таблице \ref{costs-salary}
и складываются из оплаты труда дипломника, оплаты труда руководителя дипломной работы, оплаты труда консультанта по организационно-экономическому обоснованию и оплаты труда консультанта по безопасности жизнедеятельности.

\begin{table}
\center
\caption{Издержки, связанные с оплатой труда}
\label{costs-salary}
\begin{tabular}{|p{4cm}|p{3cm}|p{2cm}|p{2.5cm}|p{2cm}|}
\hline 
 & Система оплаты & Размер оплаты & Количество & Сумма, р \\ 
\hline 
Оплата труда дипломника & повременная & 50000 р/мес & 3 месяца & 150~000 \\ 
\hline
Начисления на заработную плату &  & 30.2\% & & 45~300\\ 
\hline 
Оплата труда руководителя & повременная & 300 р/час & 20 часов & 6~000 \\ 
\hline 
Оплата труда консультанта по экономическому обоснованию & повременная & 300 р/час & 4 часа & 1200 \\ 
\hline 
Оплата труда консультанта по безопасности жизнедеятельности & повременная & 300 р/час & 2 часа & 600 \\ 
\hline 
Итого:  &  &  &  & 203~100 \\ 
\hline 
\end{tabular}
\end{table}

Издержки на транспорт сводятся к оплате проезда на общественном транспорте (метро) в течении трех месяцев, что при цене 350 рублей в месяц за льготный проездной билет дает 1050 рублей.

Издержки, связанные с работой на компьютере, указаны в таблице \ref{costs-computer},
и складываются от непосредственно издержек на работу на копьютере и издержек на 
доступ к сети интернет. Суммарно они составляют
$$ C_{к}= C_{р.к.} + C_{и} = 1500 р + 2856 р = 4365 р $$

\begin{table}
\center
\caption{Издержки, связанные с работой на компьютере и сети интернет}
\label{costs-computer}
\begin{tabular}{|p{3cm}|p{3cm}|p{3cm}|p{3cm}|}
\hline 
Тип издержек & Стоимость 1 единицы & Количество & Стоимость (р) \\ 
\hline 
Доступ к сети Интернет & 500 р/мес & 3 мес & 1500 \\ 
\hline
Работа на комрьютере & 7 р/час & 408 часов & 2856\\ 
\hline 
Итого:  &  &  & 4356 \\ 
\hline
\end{tabular}
\end{table}

Прочие издержки представлены в таблице \ref{costs-other}
и состоят из стоимости расходных материалов, электроэнергии, интернета.
При расчете расхода электроэнергии считалось, что ее мощность компьютера как 
потребителя энергии составляет 45 Вт, и работы велись 136 часов в месяц в течении
трех месяцев, т.е. суммарный расход составляет
$$ E = N t = 45 Вт \times 136 ч \times 3 = 18360 Вт \times ч \approx 18.4 кВт \times ч $$
При цене на электроэнергию в $4.02 р/кВт/ч$ получаем стоимость израсходованной электроэнергии:
$$ C_{e} = 18.4 кВт/ч \times 4.02 р/кВт/ч = 74 р $$

\begin{table}
\center
\caption{Прочие издержки}
\label{costs-other}
\begin{tabular}{|p{3cm}|p{3cm}|p{3cm}|p{3cm}|}
\hline 
Тип издержек & Стоимость 1 единицы & Количество & Стоимость (р) \\ 
\hline 
Расходы на электроэнергию & 4.02 р/кВт/ч & 18.4 квт/ч & 74 \\ 
\hline 
Печать дипломной работы & 2 р/лист & 200 листов & 400 \\ 
\hline 
Переплет дипломной работы & & & 250 руб \\
 &  &  & Итого: 724 \\ 
\hline 
\end{tabular} 
\end{table}

Таким образом, общая себестоимость данной работы составляет:
\begin{equation}
    C = C_{o} + C_{к} + C_{t} + C_{p} = 203~100 p + 4~356 р + 1~050 p + 724 p = 209~230 p \\
\end{equation}

\section{Оценка значимости дипломной работы}

Значимость дипломной работы $Д_{зн}$ вычисляется исходя из формулы:
\begin{equation}
  Д_{зн} = \frac{k_1 + k_3 + k_3 + k_4}{k_{max}}
\end{equation}
где $k_1$ описывает степень эффекта от дипломной работы, $k_2$ характеризует объем выполненных исследований и разработок, $k_3$ характеризует сложность решения в дипломной работе задачи научно-исследовательского характера, $k_4$ характеризует уровень науч-тех подготовки студента, $k_{max} = 40$.

Разработанные в ходе работы модули позволяют использовать уже существующие средства
виртуализации в виде внешнего сервиса, который может масштабироваться горизонтально. Это можно рассматривать как достижение качественно новых характеристик имеющихся средств
виртуализации. Таким образом $ k_1 = 5$.

Все модули были разработаны полностью самостоятельно, чему соответствует $k_2 = 5$.

Разработанные модули являются частью сложного программного комплекса и взаимодействуют
с большим количеством нижележащих подсистем. Для данного случая используется $k_3 = 7$.

При разработке данного программного обеспечения использовались современные языки и 
технологии, новейшие разработки в области виртуализации. Этому уровню соответствует $k_4 = 5$.

Таким образом, для данной дипломной работы значения коэфициентов равны:
\begin{eqnarray}
    k_1 &=& 5\nonumber \\
    k_2 &=& 5\nonumber \\
    k_3 &=& 7\nonumber \\
    k_4 &=& 5\nonumber 
\end{eqnarray}
что дает значение коэфициента значимости
\begin{equation}
  Д_{зн} = \frac{k_1 + k_3 + k_3 + k_4}{k_{max}} = \frac{5 + 5 + 7 + 5}{40} = 0.55
\end{equation}
\chapter{Вопросы безопасности жизнедеятельности}

Биологическое воздействие высокочастотного излучения на организм человека.


% Заключение
\conclusion


% Список литературы
\bibliography{thesis}
\bibliographystyle{gost705}

\end{document}
