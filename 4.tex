\chapter{Организационно-экономическое обоснование}

\section{Характеристика организации работ}

Разработка платформы виртуализации и моделирования сетей TCP/IP на базе OpenStack и dynamips выполнена на кафедре информационно-измерительных систем под руководством кандидата технических наук доцента Кондаурова И.\,Н. 

Исходные данные: 
\begin{itemize}
   \item исходные тексты платформы виртуализации OpenStack
   \item исходные тексты гипервизора dynamips
   \item исходные тексты модуля dynagen
   \item исходные текста программы GNS 3
\end{itemize}

Приборное и программное обеспечение:
\begin{itemize}
    \item ноутбук
    \item среда разработки IntelliJ IDEA
    \item программное обеспечение виртуализации аппаратного обеспечения маршрутизаторов, dynamips
\end{itemize}

Целью работы является создание масштабируемой архитектуры программного обеспечения, позволяющей проводить моделирование сетей TCP/IP.

Результатом работы является набор дополнительных модулей, позволяющих использовать
платформу виртуализации OpenStack для моделирования топологий сетей TCP/IP.

\section{Обоснование косвеного экономического эффекта}

Полученный в результате данной работы набор модулей может быть использован как 
непосредственно для запуска виртуализированных топологий путем подготовки специально
оформленных файлов описания, так и как часть более сложной системы, скрывающей от
пользователя конкретный синтаксис описания, в том числе с помощью графического 
интерфейса.

Построенный на базе этих модулей программно-аппаратный комплекс может быть использован
для произведения:
\begin{itemize}
    \item предварительного моделирования при создании реальной сети TCP/IP
    \item лабораторных работ, связанных с исследованием различных сетевых технологий и протоколов
    \item испытаний программного обеспечения, построенного по распределенным архитектурам, в том числе одноранговой и клиент-серверной
\end{itemize}

Стоит отметить, что все вышеуказанные способы применения данного продукта могут 
осуществляться как локально, при расположении пользователей и ИС, построенной с 
использованием разработанных модулей в пределах одного помещения, так и при 
существенном их удалении. Таким образом, результаты данной работы могут применяться 
при создании систем дистанционного обучения.

Кроме этого, одним из очевидных путей дальнейшего развития данной технологии 
является построение системы для предоставления платформы виртуализации сетей 
в виде услуги через сеть Интернет.

Использование вышеописанной платформы так же имеет положительный экономический
эффект и для конечного пользователя, позволяя арендовывать по требованию необходимые
для моделирования вычислительные ресурсы. К примеру, проведение 
предварительных испытаний топологий сетей требуется лишь на определенных этапах
разработки ИС: проведении экспериментов для выбора наилучшей топологии 
сети распределенной системы, подборе конкретных параметров перед внедрением.
Но покупка необходимого оборудования для проведения подобных испытаний
не совсем целесообразна, так как это оборудование не будет участвовать в дальнейшем
при разработке и сопровождении ИС.

\section{Расчет себестоимости дипломной работы}

Себестоимостю данной дипломной работы является сумма всех затрат, имевших место
при создании этой работы.

Себестоимость разработки дипломного проекта можно выразить в виде:
\begin{equation}
    C = C_{o} + C_{к} + C_{t} + C_{p} \\
\end{equation}
где $C$ -- общая себестоимость, $C_{к}$ -- издержки, связанные с работой на компьютере, $C_o$ -- издержки, связанные с оплатой труда, $C_t$ -- издержки на транспорт, $C_p$ -- прочие издержки.

Издержки, связанные с оплатой труда представлены в таблице \ref{costs-salary}
и складываются из оплаты труда дипломника, оплаты труда руководителя дипломной работы, оплаты труда консультанта по организационно-экономическому обоснованию и оплаты труда консультанта по безопасности жизнедеятельности.

\begin{table}
\center
\caption{Издержки, связанные с оплатой труда}
\label{costs-salary}
\begin{tabular}{|p{4cm}|p{3cm}|p{2cm}|p{2.5cm}|p{2cm}|}
\hline 
 & Система оплаты & Размер оплаты & Количество & Сумма, р \\ 
\hline 
Оплата труда дипломника & повременная & 50000 р/мес & 3 месяца & 150~000 \\ 
\hline
Начисления на заработную плату &  & 30\% & & 45~000\\ 
\hline 
Оплата труда руководителя & повременная & 300 р/час & 20 часов & 60~000 \\ 
\hline 
Оплата труда консультанта по экономическому обоснованию & повременная & 300 р/час & 4 часа & 1200 \\ 
\hline 
Оплата труда консультанта по безопасности жизнедеятельности & повременная & 300 р/час & 2 часа & 600 \\ 
\hline 
Итого:  &  &  &  & 256~800 \\ 
\hline 
\end{tabular}
\end{table}

Издержки на транспорт сводятся к оплате проезда на общественном транспорте (метро) в течении трех месяцев, что при цене 350 рублей в месяц дает 1050 рублей.

Издержки, связанные с работой на компьютере, указаны в таблице \ref{costs-computer},
и складываются от непосредственно издержек на работу на копьютере и издержек на 
доступ к сети интернет. Суммарно они составляют
$$ C_{к}= C_{р.к.} + C_{и} = 1500 р + 2856 р = 4365 р $$

\begin{table}
\center
\caption{Издержки, связанные с работой на компьютере}
\label{costs-computer}
\begin{tabular}{|p{3cm}|p{3cm}|p{3cm}|p{3cm}|}
\hline 
Тип издержек & Стоимость 1 единицы & Количество & Стоимость (р) \\ 
\hline 
Доступ к сети Интернет & 500 р/мес & 3 мес & 1500 \\ 
\hline
Работа на комрьютере & 7 р/час & 408 часов & 2856\\ 
\hline 
Итого:  &  &  & 4356 \\ 
\hline
\end{tabular}
\end{table}

Прочие издержки представлены в таблице \ref{costs-other}
и состоят из стоимости расходных материалов, электроэнергии, интернета.
При расчете расхода электроэнергии считалось, что ее мощность компьютера как 
потребителя энергии составляет 45 Вт, и работы велись 136 часов в месяц в течении
трех месяцев, т.е. суммарный расход составляет
$$ E = N t = 45 Вт \times 136 ч \times 3 = 18360 Вт \times ч \approx 18.4 кВт \times ч $$

\begin{table}
\center
\caption{Прочие издержки}
\label{costs-other}
\begin{tabular}{|p{3cm}|p{3cm}|p{3cm}|p{3cm}|}
\hline 
Тип издержек & Стоимость 1 единицы & Количество & Стоимость (р) \\ 
\hline 
Расходы на электроэнергию & 4.02 р/кВт/ч & 18.4 квт/ч & 77 \\ 
\hline 
Печать & 2 р/лист & 200 листов & 400 \\ 
\hline 
 &  &  & Итого: 477 \\ 
\hline 
\end{tabular} 
\end{table}

Таким образом, общая себестоимость данной работы составляет:
\begin{equation}
    C = C_{o} + C_{к} + C_{t} + C_{p} = 256~800 p + 4~356 р + 1~050 p + 477 p = 262~683 p \\
\end{equation}

\section{Оценка значимости дипломной работы}

Значимость дипломной работы $Д_{зн}$ вычисляется исходя из формулы:
\begin{equation}
  Д_{зн} = \frac{k_1 + k_3 + k_3 + k_4}{k_{max}}
\end{equation}
где $k_1$ описывает степень эффекта от дипломной работы, $k_2$ характеризует объем выполненных исследований и разработок, $k_3$ характеризует сложность решения в дипломной работе задачи научно-исследовательского характера, $k_4$ характеризует уровень науч-тех подготовки студента, $k_{max} = 40$.

Разработанные в ходе работы модули позволяют использовать уже существующие средства
виртуализации в виде внешнего сервиса, который может масштабироваться горизонтально. Это можно рассматривать как достижение качественно новых характеристик имеющихся средств
виртуализации. Таким образом $ k_1 = 5$.

Все модули были разработаны полностью самостоятельно, чему соответствует $k_2 = 5$.

Разработанные модули являются частью сложного программного комплекса, взаимодействуя
с большим количеством нижележащих подсистем. Для данного случая используется $k_3 = 7$.

При разработке данного программного обеспечения использовались современные языки и 
технологии, новейшие разработки в области виртуализации. Этому уровню соответствует $k_4 = 5$.

Таким образом, для данной дипломной работы значения коэфициентов равны:
\begin{eqnarray}
    k_1 &=& 5\nonumber \\
    k_2 &=& 5\nonumber \\
    k_3 &=& 7\nonumber \\
    k_4 &=& 5\nonumber 
\end{eqnarray}
что дает значение коэфициента значимости
\begin{equation}
  Д_{зн} = \frac{k_1 + k_3 + k_3 + k_4}{k_{max}} = \frac{5 + 5 + 7 + 5}{40} = 0.55
\end{equation}