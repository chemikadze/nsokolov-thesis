\conclusion

В результате данной работы было проверено и подтверждено предположение 
о возможности применения технологий виртуализации при моделировании сетей.
Были рассмотрены требования к подобной системе и возможные варианты 
применения, такие как образование и проектирование. В результате было
получено понимание о структуре подобной системы.

Как подтверждение концепции, был разработан набор модулей, который 
позволяет использовать хорошо зарекомендовавшую себя платформу
виртуализации OpenStack в качестве основы системы моделирования.
Кроме этого, получившееся реализация может рассматриваться как подтверждение
гибкости использованной платформы при применении в экспериментальных
разработках, связанных с облачными вычислениями.

Положительной особенностью созданной системы является возможность
использования не только непосредственно, но и как основы для создания
учебных и исследовательских сред моделирования. Благодаря тому, что
интерфейс системы базируется на простых и хорошо зарекомендовавших 
себя стандартах, создание подобных надстроек не должно составлять
большого труда.

Наиболее приоритетным с точки зрения автора является дальнейшее развитие
именно в сторону применения в учебном процессе, так как далеко немногие
образовательные учреждения, проводящие обучение сетевым технологиям,
имеют доступ к реальному оборудованию. Кроме этого, вынесение 
ресурсоемких вычислений стены учебных заведений позволяет расширить возможности
получения образования тех учащихся, которые в силу тех или иных причин не 
имеют прямого доступа  к большим вычислительным ресурсам.

Что касается развития самой системы, то она так же имеет потенциал для дальнейшего
развития. По мере устаревания оборудования, которое моделируется при помощи
использованного гипервизора сетевых устройств, возникает необходимость в поддержке
новых типов оборудования, в особенности набирающих в последнее время популярность
программных коммутаторов. 

Так же очень перспективным направлением является индивидуальная подстройка
параметров каналов прямо во время работы запущенной топологии. Подобная 
функциональность позволила бы моделировать поведение информационных систем во 
время различного рода сбоев. Сюда же можно отнести улучшение гранулярности
управления устройствами. При одновременном развитии в двух указанных выше 
направлениях, учебном и исследовательском, система может стать гибкой
и универсальной средой.

В заключение хотелось бы добавить, что данная область является сравнительно молодой --
первые промышленные образцы систем подобного рода появились в середине
2000-х, и стремительный прогресс в этой области все еще продолжается. Остается
только гадать, какие еще неожиданные применения облачным технологиям появятся в будущем.

